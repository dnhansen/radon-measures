% Document setup
\documentclass[article, a4paper, 11pt, oneside]{memoir}
\usepackage[utf8]{inputenc}
\usepackage[T1]{fontenc}
\usepackage[UKenglish]{babel}

% Document info
\newcommand\doctitle{Radon Measures}
\newcommand\docauthor{Danny Nygård Hansen}

% Formatting and layout
\usepackage[autostyle]{csquotes}
\usepackage[final]{microtype}
\usepackage{xcolor}
\frenchspacing
\usepackage{latex-sty/articlepagestyle}
\usepackage{latex-sty/articlesectionstyle}

% Fonts
\usepackage[largesmallcaps,partialup]{kpfonts}
\DeclareSymbolFontAlphabet{\mathrm}{operators} % https://tex.stackexchange.com/questions/40874/kpfonts-siunitx-and-math-alphabets
\linespread{1.06}
\let\mathfrak\undefined
\usepackage{eufrak}
\usepackage{inconsolata}
\usepackage{amssymb}

% Hyperlinks
\usepackage{hyperref}
\definecolor{linkcolor}{HTML}{4f4fa3}
\hypersetup{%
	pdftitle=\doctitle,
	pdfauthor=\docauthor,
	colorlinks,
	linkcolor=linkcolor,
	citecolor=linkcolor,
	urlcolor=linkcolor,
	bookmarksnumbered=true
}

% Equation numbering
\numberwithin{equation}{chapter}

% Footnotes
\footmarkstyle{\textsuperscript{#1}\hspace{0.25em}}

% Mathematics
\usepackage{latex-sty/basicmathcommands}
\usepackage{latex-sty/framedtheorems}
\usepackage{latex-sty/topologycommands}
\usepackage{tikz-cd}
\usetikzlibrary{babel}

% Lists
\usepackage{enumitem}
\setenumerate[0]{label=\normalfont(\arabic*)}

% Bibliography
\usepackage[backend=biber, style=authoryear, maxcitenames=2, useprefix]{biblatex}
\addbibresource{references.bib}

% Title
\title{\doctitle}
\author{\docauthor}


% Section style -- add to section style .sty?
\setsubsecheadstyle{\normalfont\itshape}


% Preimage -- to be added to mathcommands .sty
\newcommand{\preim}{^{-1}}


\newcommand{\calN}{\mathcal{N}}
\DeclarePairedDelimiter{\nhoodfilteraux}{(}{)}
% \newcommand{\nhoodfilter}[1]{\calN\nhoodfilteraux{#1}}
\newcommand{\nhoodfilter}[1]{\calN_{#1}}


\newcommand{\calU}{\mathcal{U}}
\newcommand{\calV}{\mathcal{V}}
\newcommand{\calW}{\mathcal{W}}
\newcommand{\calT}{\mathcal{T}}
\newcommand{\calB}{\mathcal{B}}
\newcommand{\calE}{\mathcal{E}}

\newcommand{\borel}[1]{\calB(#1)}
\DeclareMathOperator{\supp}{supp}

\begin{document}

\maketitle

\chapter{Introduction}


\chapter{General properties of measures}

We assume that the reader is familiar with abstract measure spaces and topological spaces. Below we fix terminology and prove some elementary results.


\section{Essential and principal measures}

\newcommand{\calJ}{\mathcal{J}}
\newcommand{\powerset}[1]{2^{#1}}

If $\calJ \subseteq \powerset{X}$ and $\mu \colon \calJ \to [0,\infty]$ such that $\emptyset \in \calJ$ and $\mu(\emptyset) = 0$, then $\mu$ gives rise to an outer measure $\mu^*$ on $X$ by
%
\begin{equation*}
    \mu^*(A)
        = \inf \set[\Big]{
            \sum_{n=1}^\infty \mu(B_n)
        }{
            (B_n)_{n \in \naturals} \subseteq \calJ
            \text{ and }
            A \subseteq \bigunion_{n \in \naturals} B_n
        }.
\end{equation*}
%
In the case that $\mu$ is a measure and $\calJ$ is a $\sigma$-algebra, we may rephrase this as
%
\begin{equation*}
    \mu^*(A)
        = \inf \set{
            \mu(B)
        }{
            B \in \calJ
            \text{ and }
            A \subseteq B
        }.
\end{equation*}
%
In this case we also have $\mu^*(A) = \mu(A)$ if $A \in \calJ$.

\begin{definition}[Essential measures]
    Let $(X, \calE, \mu)$ be a measure space. The \emph{essential measure} associated with $\mu$ is the map $\nu \colon \borel{X} \to [0,\infty]$ given by
    %
    \begin{equation}
        \label{eq:essential-measure-definition}
        \nu(A)
            = \sup \set{ \mu^*(B) }{ B \subseteq A \text{ and } \mu^*(B) < \infty },
    \end{equation}
    %
    for all $A \in \calE$.
\end{definition}


\begin{lemma}
    \label{thm:essential-measure-equivalent-formula}
    If $(X,\calE,\mu)$ is a measure space and $\nu$ is the essential measure associated with $\mu$, then $\nu(A) = \mu(A)$ when $\mu(A) < \infty$ or $\nu(A) = \infty$, and
    %
    \begin{equation}
        \label{eq:essential-measure-equivalent-formula}
        \nu(A)
            = \sup \set{ \mu(B) }{ B \in \calE, B \subseteq A \text{ and } \mu(B) < \infty },
    \end{equation}
    %
    for all $A \in \calE$. Furthermore, $\nu$ is a measure on $\calE$.
\end{lemma}
%
Given this lemma, I have no idea why one would define essential measures by \eqref{eq:essential-measure-definition}, but this is the way it is done in \textcite{schwartz1973} and \textcite{nlab:radon_measure}.

\begin{proof}
    Let $A \in \calE$, and assume that $\mu(A) < \infty$. Then $\mu^*(A) = \mu(A) < \infty$ by the definition of the outer measure $\mu^*$, so $\nu(A) = \mu^*(A)$ by the definition of $\nu$. With this, \eqref{eq:essential-measure-equivalent-formula} is obvious when $\mu(A) < \infty$.

    Now assume that $\nu(A) = \infty$. Then for any $R > 0$ there exists a $B \subseteq A$ such that $\mu^*(B) \geq R$. Now let $C \in \calE$ such that $B \subseteq C$ and $\mu(C) < \infty$. Then $B \subseteq A \intersect C \subseteq A$, so
    %
    \begin{equation*}
        R
            \leq \mu^*(B)
            \leq \mu(A \intersect C)
            < \infty.
    \end{equation*}
    %
    Since $R$ was arbitrary, $\mu$ can take on arbitrarily large but finite values on subsets of $A$, so \eqref{eq:essential-measure-equivalent-formula} follows. It also follows that $\mu(A) = \infty$.

    Finally we show that $\nu$ is a Borel measure on $X$. Let $(A_n)_{n \in \naturals}$ be a sequence of pairwise disjoint sets in $\calE$, and let $B \subseteq \bigunion_{n \in \naturals} A_n$ be such that $\mu(B) < \infty$. If we let $B_n = B \intersect A_n$, then
    %
    \begin{equation*}
        \mu(B)
            = \mu \Bigl( \bigunion_{n \in \naturals} B_n \Bigr)
            = \sum_{n=1}^\infty \mu(B_n)
            \leq \sum_{n=1}^\infty \nu(A_n),
    \end{equation*}
    %
    where the inequality follows since $B_n \subseteq A_n$ and $\mu(B_n) < \infty$. This inequality holds for all $B \subseteq \bigunion_{n \in \naturals} A_n$, so taking the supremum over such $B$ yields
    %
    \begin{equation*}
        \nu \Bigl( \bigunion_{n \in \naturals} A_n \Bigr)
            \leq \sum_{n=1}^\infty \nu(A_n).
    \end{equation*}
    %
    We prove the opposite inequality. If the left-hand side is infinite this is obvious, so assume that it is finite. For $\epsilon > 0$ there is then a sequence $(C_n)_{n \in \naturals}$ in $\calE$ with $C_n \subseteq A_n$ and $\mu(C_n) < \infty$ such that $\nu(A_n) \leq \mu(C_n) + \epsilon/2^n$. By continuity of $\mu$ we have
    %
    \begin{equation*}
        \mu \Bigl( \bigunion_{n \in \naturals} C_n \Bigr)
            = \lim_{k \to \infty} \mu \Bigl( \bigunion_{n \leq k} C_n \Bigr)
            \leq \nu \Bigl( \bigunion_{n \in \naturals} A_n \Bigr)
            < \infty,
    \end{equation*}
    %
    since $\mu(C_n) \leq \nu(A_n)$. Notice that the $C_n$ are also pairwise disjoint, so it follows that
    %
    \begin{equation*}
        \sum_{n=1}^\infty \nu(A_n) - \epsilon
            \leq \sum_{n=1}^\infty \mu(C_n)
            = \mu \Bigl( \bigunion_{n \in \naturals} C_n \Bigr)
            \leq \nu \Bigl( \bigunion_{n \in \naturals} A_n \Bigr),
    \end{equation*}
    %
    and since $\epsilon$ was arbitrary, the inequality follows.
\end{proof}

We take the first step towards constructing an \textquote{inverse} to taking the essential measure. I am not sure that this is possible in general, and we shall only be interested in this issue in the context of Radon measures. However, we give the definition of this \textquote{inverse} operation since it does not require any further developments.

If $X$ is a topological space, we denote the Borel algebra on $X$ by $\borel{X}$. A \emph{Borel measure} on $X$ is then a measure on $\borel{X}$.

\begin{definition}[Principal measures]
    Let $\nu$ be a Borel measure on a topological space $X$. The \emph{principal measure}\footnotemark{} associated with $\nu$ is the map $\mu \colon \borel{X} \to [0,\infty]$ given by
    %
    \begin{equation*}
        \mu(A)
            = \inf \set{\nu(U)}{\text{$U$ open and $A \subseteq U$}},
    \end{equation*}
    %
    for all $A \in \borel{X}$.
\end{definition}
\footnotetext{I do not believe this is standard terminology. \textcite{bauer2001} uses the term \textquote{principal measure} for a certain measure determined by a positive linear functional, as given by the Riesz representation theorem. This measure is what we will call an \emph{outer Radon measure} in \cref{def:Radon-measures}.}
%
Whether this is actually a measure in general I don't know, but it does not seem to be. In \cref{thm:Radon-pair-inverses} we shall prove that this is in fact the case when $\mu$ is what we will term an outer Radon measure. The proof seems to make essential use of this fact, so it seems unlikely that the principal measure associated with an arbitrary Borel measure is in fact a measure. Perhaps it is possible to define a different kind of map, replacing the assumption that $U$ is open with $U$ simply being measurable, perhaps with respect to a $\sigma$-algebra different from the Borel algebra. We will not pursue this further.


\section{Borel measures on Hausdorff spaces}

Below we let $X$ denote a Hausdorff topological space. A Borel measure $\mu$ on $X$ is called \emph{outer regular} on a set $B \in \borel{X}$ if
%
\begin{equation*}
    \mu(B)
        = \inf \set{\mu(U)}{U \supseteq B, \text{$U$ open}},
\end{equation*}
%
and \emph{inner regular} on $B$ if
%
\begin{equation*}
    \mu(B)
        = \sup \set{\mu(K)}{K \subseteq B, \text{$K$ compact}}.
\end{equation*}
%
If $\mu$ is outer (inner) regular on all Borel sets, then we call it \emph{outer (inner) regular}. Furthermore, if $\mu$ is both outer and inner regular, then it is simply called \emph{regular}.

A Borel measure $\mu$ on $X$ is called \emph{locally finite} if every point has a neighbourhood $U$ with $\mu(U) < \infty$. We have the following characterisation of local finiteness:

\begin{proposition}
    \label{thm:local-finiteness-compacts}
    If a Borel measure on $X$ is locally finite, then it is finite on all compact sets. The converse is also true if $X$ is locally compact.
\end{proposition}

\begin{proof}
    Let $\mu$ be a locally finite Borel measure on $X$, and let $K \subseteq X$ be compact. Every $x \in K$ has an open neighbourhood $U_x$ with $\mu(U_x) < \infty$. The collection $\set{U_x}{x \in K}$ is an open cover of $K$, so it has a finite subcover, say $U_{x_1}, \ldots, U_{x_n}$. But then
    %
    \begin{equation*}
        \mu(K)
            \leq \mu \Bigl( \bigunion_{i=1}^n U_{x_i} \Bigr)
            \leq \sum_{i=1}^n \mu(U_{x_i})
            < \infty,
    \end{equation*}
    %
    as desired.

    Conversely, suppose that $X$ is locally compact and that $\mu$ is a Borel measure on $X$ that is finite on compact sets. Then every point has a compact neighbourhood, so every point has a neighbourhood on which $\mu$ is finite. Hence $\mu$ is locally finite.
\end{proof}


\chapter{Radon measures}

\section{Definitions and basic properties}

Radon measures are usually only defined on \emph{locally compact} Hausdorff spaces, but this extra assumption is for many purposes superfluous and overly strict: Indeed, another natural setting for Radon measures is that of Polish spaces (or more general metrisable spaces), particularly in probability theory. Thus we consider a general Hausdorff topological space $X$ below.

\newcommand{\calM}{\mathcal{M}}
\newcommand{\radonout}{\calM^{+}}
\newcommand{\radonin}{\calM^{-}}

\begin{definition}[Radon measures]
    \label{def:Radon-measures}
    An \emph{outer Radon measure} on $X$ is a Borel measure on $X$ that is locally finite, outer regular, and inner regular on open sets.

    An \emph{inner Radon measure} on $X$ is a Borel measure on $X$ that is locally finite and inner regular.
\end{definition}
%
\textcite{schwartz1973} gives three different (but equivalent, as he shows) definitions of Radon measures. His definition $R_2$ is what we call an outer Radon measure, and his $R_3$ is our inner Radon measure. We will return to his $R_1$ definition in \cref{def:Radon-pairs}.

We begin by deriving some properties of Radon measures.


\begin{proposition}
    \label{thm:outer-Radon-inner-regular-on-finites}
    Every outer Radon measure is inner regular on all its $\sigma$-finite sets.
\end{proposition}
%
This is Proposition~7.5 in \textcite{folland2007}, though Folland only considers Radon measures on locally compact spaces. In fact the claim holds in any Hausdorff space, as the proof below demonstrates.

\begin{proof}
    Let $\mu$ be an outer Radon measure on $X$, and let $A \in \borel{X}$. First assume that $\mu(A) < \infty$ and let $\epsilon > 0$. By outer regularity there is an open set $U \supseteq A$ such that $\mu(U) < \mu(A) + \epsilon$, and by inner regularity on $U$ there is a compact set $K \subseteq U$ with $\mu(U) < \mu(K) + \epsilon$. Furthermore, since $\mu(U \setminus A) < \epsilon$ there exists an open set $V \supseteq U \setminus A$ such that $\mu(V) < \epsilon$. Now let $F = K \setminus V$ and notice that $F$ is compact and that $F \subseteq A$. It follows that
    %
    \begin{equation*}
        \mu(F)
            = \mu(K) - \mu(K \intersect V)
            > \mu(A) - \epsilon - \mu(V)
            > \mu(A) - 2\epsilon.
    \end{equation*}
    %
    Hence $\mu$ is inner regular on $A$.

    Now assume that $\mu(A) = \infty$ and that there exists an increasing sequence $(A_n)_{n \in \naturals}$ in $\borel{X}$ with $\mu(A_n) < \infty$, and whose union is $A$. It follows by continuity of $\mu$ that $\mu(A_n) \to \infty$ as $n \to \infty$, so for any $R > 0$ there is an $n \in \naturals$ such that $\mu(A_n) > R$. By inner regularity on $A_n$ there exists a compact set $K \subset A_n$ such that $\mu(K) > R$. Hence $\mu$ is also inner regular on $A$.
\end{proof}

\begin{lemma}
    \label{thm:Radon-agree-on-open-sets}
    Let $\mu$ be an outer Radon measure on $X$, and let $\nu$ be the essential measure associated with $\mu$. Then $\nu(U) = \mu(U)$ for all open $U \subseteq X$.
\end{lemma}

\begin{proof}
    Let $U \subseteq X$ be open. Since $\nu \leq \mu$ in general, we only need to show that $\nu(U) \geq \mu(U)$. Since $\mu$ is inner regular on $U$, there exists a sequence $(K_n)_{n \in \naturals}$ of compact subsets of $U$ such that $\mu(U) = \lim_{n \to \infty} \mu(K_n)$. Furthermore, $\mu(K_n) < \infty$ because $\mu$ is locally finite, so since $\nu$ and $\mu$ agree when $\mu$ is finite it follows that
    %
    \begin{equation*}
        \mu(U)
            = \lim_{n \to \infty} \mu(K_n)
            = \lim_{n \to \infty} \nu(K_n)
            \leq \nu(U),
    \end{equation*}
    %
    as desired.
\end{proof}

% Add to framedtheorems.sty
\newcommand{\mylistlabelfont}[1]{{\normalfont\color{linkcolor}\textit{#1}:}}
\newlist{proofsec}{description}{1}
\setlist[proofsec]{leftmargin=0pt, parsep=0pt, listparindent=\parindent, font=\mylistlabelfont}

We denote the set of outer Radon measures on $X$ by $\radonout(X)$ and the set of inner Radon measures by $\radonin(X)$. Define maps $E \colon \radonout(X) \to \radonin(X)$ and $P \colon \radonin(X) \to \radonout(X)$, where $E$ maps an outer Radon measure to the corresponding essential measure, and $P$ maps an inner Radon measure to the corresponding principal measure.

\begin{theorem}
    \label{thm:Radon-pair-inverses}
    The maps $E$ and $P$ are well-defined and each other's inverses.
\end{theorem}
% Maybe pull some things out as lemmas.

\begin{proof}
\begin{proofsec}
    \item[Well-definition of $E$]
    Let $\mu$ be an outer Radon measure on $X$, and let $\nu$ be the associated essential measure. If $A \in \borel{X}$ and $\mu(A) < \infty$, then $\mu$ is inner regular on $A$ by \cref{thm:outer-Radon-inner-regular-on-finites}. But since $\mu$ and $\nu$ agree on sets with finite $\mu$-measure, $\nu$ is also inner regular on $A$.

    Similarly, every point in $X$ has a neighbourhood with finite $\mu$-measure, so this neighbourhood also has finite $\nu$-measure. Thus $\nu$ is an inner Radon measure.

    \item[Well-definition of $P$]
    Let $\nu$ be an inner Radon measure on $X$ with associated principal measure $\mu$. We first show that $\mu$ and $\nu$ agree on $\mu$-finite sets, so let $A \in \borel{X}$ with $\mu(A) < \infty$. Clearly $\nu(A) \leq \mu(A)$, so we prove the other inequality. Let $U \supseteq A$ be an open set with $\nu(U) < \infty$. Then also $\nu(U \setminus A) < \infty$, so for $\epsilon > 0$ there exists a compact set $K \subseteq U \setminus A$ with $\nu(U \setminus A) \leq \nu(K) + \epsilon$ by inner regularity. Then $V = U \setminus K$ is an open set containing $A$, so
    %
    \begin{equation*}
        \mu(A)
            \leq \nu(V)
            = \nu(A) + \nu(U \setminus A) - \nu(K)
            \leq \nu(A) + \epsilon.
    \end{equation*}
    %
    Since $\epsilon$ was arbitrary, it follows that $\mu(A) \leq \nu(A)$.

    Next we show that $\mu$ is $\sigma$-additive, so let $(A_n)_{n \in \naturals}$ be a sequence of disjoints sets in $\borel{X}$. For $\epsilon > 0$ there exists a sequence $(U_n)_{n \in \naturals}$ of open sets with $A_n \subseteq U_n$ such that $\nu(U_n) \leq \mu(A_n) + \epsilon$. It follows that
    %
    \begin{equation*}
        \mu \Bigl( \bigunion_{n \in \naturals} A_n \Bigr)
            \leq \nu \Bigl( \bigunion_{n \in \naturals} U_n \Bigr)
            = \sum_{n=1}^\infty \nu(U_n)
            \leq \sum_{n=1}^\infty \mu(A_n) + \epsilon.
    \end{equation*}
    %
    So $\mu$ is countably subadditive since $\epsilon$ was arbitrary. The opposite inequality is obvious if $\mu( \bigunion_{n \in \naturals} A_n ) = \infty$, and if not then the sets $A_n$ also have finite $\mu$-measure. Hence
    %
    \begin{equation*}
        \mu \Bigl( \bigunion_{n \in \naturals} A_n \Bigr)
            = \nu \Bigl( \bigunion_{n \in \naturals} A_n \Bigr)
            = \sum_{n=1}^\infty \nu(A_n)
            = \sum_{n=1}^\infty \mu(A_n).
    \end{equation*}
    %
    Thus $\mu$ is in fact a measure.

    Finally, $\mu$ is clearly locally finite since $\nu$ is, and since $\mu$ and $\nu$ agree on $\mu$-finite sets, outer regularity follows easily from the definition of $\mu$.

    \item[$P \circ E = \id$]
    Let $\mu$ be an outer Radon measure, $\nu = E(\mu)$ its corresponding essential measure, and $\mu' = P(\nu)$ the principal measure associated with $\nu$. We must show that $\mu = \mu'$.

    Let $A \in \borel{X}$. Since $\mu$ is outer regular,
    %
    \begin{equation*}
        \mu(A)
            = \inf \set{\mu(U)}{\text{$U$ open and $A \subseteq U$}}.
    \end{equation*}
    %
    Comparing this with the definition of $\mu'$ and recalling that $\nu \leq \mu$, we find that $\mu'(A) \leq \mu(A)$.

    For the opposite inequality, let $\epsilon > 0$ and let $U \supseteq A$ be an open set such that $\nu(U) \leq \mu'(A) + \epsilon$. Because $\nu(U) = \mu(U)$ by \cref{thm:Radon-agree-on-open-sets}, we have
    %
    \begin{equation*}
        \mu'(A) + \epsilon
            \geq \nu(U)
            = \mu(U)
            \geq \mu(A),
    \end{equation*}
    %
    so it follows that $\mu'(A) \geq \mu(A)$ since $\epsilon$ was arbitrary.

    \item[$E \circ P = \id$]
    Conversely, let $\nu$ be an inner Radon measure and let $\mu = P(\nu)$ and $\nu' = E(\mu)$. Let $A \in \borel{X}$ and notice that
    %
    \begin{align*}
        \nu(A)
            &= \sup \set{\nu(K)}{ \text{$K$ compact and $K \subseteq A$} } \\
            &= \sup \set{\mu(K)}{ \text{$K$ compact and $K \subseteq A$} } \\
            &= \sup \set{\mu(B)}{ \text{$B \in \borel{X}$, $B \subseteq A$ and $\mu(B) < \infty$} } \\
            &= \nu'(A).
    \end{align*}
    %
    The second and third equalities follow since $\mu$ is locally finite, hence finite on compact sets by \cref{thm:local-finiteness-compacts}, so $\nu(K) = \mu(K) < \infty$.
\end{proofsec}
\end{proof}


\begin{proposition}
    If $\mu$ is an outer Radon measure that is also inner regular, then $E(\mu) = \mu$. Conversely, if $\nu$ is an inner Radon measure that is also outer regular, then $P(\nu) = \nu$.
\end{proposition}

\begin{proof}
    Assume that $\mu$ is an outer Radon measure, let $\nu = E(\mu)$, and let $A \in \borel{X}$. Then $\nu \leq \mu$ with equality on $\mu$-finite sets, so it suffices to prove that $\nu(A) = \infty$ if $\mu(A) = \infty$. In this case there is for any $R > 0$ a compact set $K \subseteq A$ such that $\mu(R) > R$ by inner regularity. But $\mu$ is locally finite, so $\mu(K) < \infty$ and hence $\nu(A) > R$. This holds for all $R$, so $\nu(A) = \infty$, proving the claim.

    Now assume that $\nu$ is an inner Radon measure. Let $\mu = P(\nu)$, and consider $A \in \borel{X}$. As before, we only need to consider the case where $\mu(A) = \infty$. Then $\nu(U) = \infty$ for all open $U \supseteq A$. But outer regularity of $\nu$ then implies that also $\nu(A) = \infty$.
\end{proof}




\begin{definition}[Radon pairs]
    \label{def:Radon-pairs}
    Let $\mu$ and $\nu$ be outer and inner Radon measures respectively, such that $\nu = E(\mu)$ (or equivalently $\mu = P(\nu)$). The pair $(\mu,\nu)$ is then called a \emph{Radon pair}\footnotemark{}.
\end{definition}
\footnotetext{Similar to principal measures, I do not believe this is standard terminology. In fact, as far as I know I have just made it up!}
%
A Radon pair is what \textcite{schwartz1973} describes in his $R_1$ definition of Radon measures.


\section{The Riesz Representation Theorem}

Now let $X$ be a locally compact Hausdorff space, and let $C_c(X)$ denote the space of continuous complex-valued functions on $X$. A linear functional $I$ on $C_c(X)$ is said to be \emph{positive} if $I(f) \geq 0$ when $f \geq 0$. A Borel measure $\mu$ on $X$ is called a \emph{representing measure} for $I$ if $I(f) = \int f \dif\mu$ for all $f \in C_c(X)$.

\begin{theorem}[The Riesz Representation Theorem]
    Every positive linear functional on $C_c(X)$ has a unique outer Radon representing measure.
\end{theorem}

\begin{proof}
    \textcite[Theorem~7.2]{folland2007}.
\end{proof}

\begin{proposition}
    Let $(\mu,\nu)$ be a Radon pair on $X$, and let $I$ be a positive linear functional on $C_c(X)$. Then $\mu$ is a representing measure for $I$ if and only if $\nu$ is. [Uniqueness?]
\end{proposition}

\begin{proof}
    This amounts to showing that
    %
    \begin{equation*}
        \int f \dif \nu = \int f \dif \mu
    \end{equation*}
    %
    for all $f \in C_c(X)$. Pick one such $f$, and let $K = \supp f$. Since $K$ is compact and both $\nu$ and $\mu$ are locally finite, \cref{thm:essential-measure-equivalent-formula} implies that $\nu$ and $\mu$ agree when restricted to $K$. The claim follows.
\end{proof}


\nocite{*}

\printbibliography


\end{document}