% Document setup
\documentclass[article, a4paper, 11pt, oneside]{memoir}
\usepackage[utf8]{inputenc}
\usepackage[T1]{fontenc}
\usepackage[UKenglish]{babel}

% Document info
\newcommand\doctitle{Radon Measures}
\newcommand\docauthor{Danny Nygård Hansen}

% Formatting and layout
\usepackage[autostyle]{csquotes}
\usepackage[final]{microtype}
\usepackage{xcolor}
\frenchspacing
\usepackage{latex-sty/articlepagestyle}
\usepackage{latex-sty/articlesectionstyle}

% Fonts
\usepackage[largesmallcaps,partialup]{kpfonts}
\DeclareSymbolFontAlphabet{\mathrm}{operators} % https://tex.stackexchange.com/questions/40874/kpfonts-siunitx-and-math-alphabets
\linespread{1.06}
\let\mathfrak\undefined
\usepackage{eufrak}
\usepackage{inconsolata}
\usepackage{amssymb}

% Hyperlinks
\usepackage{hyperref}
\definecolor{linkcolor}{HTML}{4f4fa3}
\hypersetup{%
	pdftitle=\doctitle,
	pdfauthor=\docauthor,
	colorlinks,
	linkcolor=linkcolor,
	citecolor=linkcolor,
	urlcolor=linkcolor,
	bookmarksnumbered=true
}

% Equation numbering
\numberwithin{equation}{chapter}

% Footnotes
\footmarkstyle{\textsuperscript{#1}\hspace{0.25em}}

% Mathematics
\usepackage{latex-sty/basicmathcommands}
\usepackage{latex-sty/framedtheorems}
\usepackage{latex-sty/topologycommands}
\usepackage{tikz-cd}
\usetikzlibrary{babel}

% Lists
\usepackage{enumitem}
\setenumerate[0]{label=\normalfont(\arabic*)}

% Bibliography
\usepackage[backend=biber, style=authoryear, maxcitenames=2, useprefix]{biblatex}
\addbibresource{references.bib}

% Title
\title{\doctitle}
\author{\docauthor}


% Section style -- add to section style .sty?
\setsubsecheadstyle{\normalfont\itshape}


% Preimage -- to be added to mathcommands .sty
\newcommand{\preim}{^{-1}}


\newcommand{\calN}{\mathcal{N}}
\DeclarePairedDelimiter{\nhoodfilteraux}{(}{)}
% \newcommand{\nhoodfilter}[1]{\calN\nhoodfilteraux{#1}}
\newcommand{\nhoodfilter}[1]{\calN_{#1}}


\newcommand{\calU}{\mathcal{U}}
\newcommand{\calV}{\mathcal{V}}
\newcommand{\calW}{\mathcal{W}}
\newcommand{\calT}{\mathcal{T}}
\newcommand{\calB}{\mathcal{B}}

\newcommand{\borel}[1]{\calB(#1)}

\begin{document}

\maketitle

\chapter{Introduction}


\chapter{General properties of measures}

We assume that the reader is familiar with abstract measure spaces and topological spaces. Below we fix terminology and prove some elementary results.

Below we let $X$ denote a Hausdorff topological space. A \emph{Borel measure} on $X$ is a measure on the Borel $\sigma$-algebra $\borel{X}$ of $X$. A Borel measure $\mu$ on $X$ is called \emph{outer regular} on a set $B \in \borel{X}$ if
%
\begin{equation*}
    \mu(B)
        = \inf \set{\mu(U)}{U \supseteq B, \text{$U$ open}},
\end{equation*}
%
and \emph{inner regular} on $B$ if
%
\begin{equation*}
    \mu(B)
        = \sup \set{\mu(K)}{K \subseteq B, \text{$K$ compact}}.
\end{equation*}
%
If $\mu$ is outer (inner) regular on all Borel sets, then we call it \emph{outer (inner) regular}. If $\mu$ is both outer and inner regular, then it is simply called \emph{regular}.

A Borel measure $\mu$ on $X$ is called \emph{locally finite} if every point has a neighbourhood $U$ with $\mu(U) < \infty$. We have the following characterisation of local finiteness:

\begin{proposition}
    If a Borel measure on $X$ is locally finite, then it is finite on all compact sets. The converse is also true if $X$ is locally compact.
\end{proposition}

\begin{proof}
    Let $\mu$ be a locally finite Borel measure on $X$, and let $K \subseteq X$ be compact. Every $x \in K$ has an open neighbourhood $U_x$ with $\mu(U_x) < \infty$. The collection $\set{U_x}{x \in K}$ is an open cover of $K$, so it has a finite subcover, say $U_{x_1}, \ldots, U_{x_n}$. But then
    %
    \begin{equation*}
        \mu(K)
            \leq \mu \Bigl( \bigunion_{i=1}^n U_{x_i} \Bigr)
            \leq \sum_{i=1}^n \mu(U_{x_i})
            < \infty,
    \end{equation*}
    %
    as desired.

    Conversely, suppose that $X$ is locally compact and that $\mu$ is a Borel measure that is finite on compact sets. Then every point has a compact neighbourhood, so every point has a neighbourhood on which $\mu$ is finite. Hence $\mu$ is locally finite.
\end{proof}

\newcommand{\calJ}{\mathcal{J}}
\newcommand{\powerset}[1]{2^{#1}}

If $\calJ \subseteq \powerset{X}$ and $\mu \colon \calJ \to [0,\infty]$ such that $\emptyset \in \calJ$ and $\mu(\emptyset) = 0$, then $\mu$ gives rise to an outer measure $\mu^*$ on $X$ by
%
\begin{equation*}
    \mu^*(A)
        = \inf \set[\Big]{
            \sum_{n=1}^\infty \mu(B_n)
        }{
            (B_n)_{n \in \naturals} \subseteq \calJ
            \text{ and }
            A \subseteq \bigunion_{n \in \naturals} B_n
        }.
\end{equation*}
%
In the case that $\mu$ is a measure and $\calJ$ is a $\sigma$-algebra, we may rephrase this as
%
\begin{equation*}
    \mu^*(A)
        = \inf \set{
            \mu(B)
        }{
            B \in \calJ
            \text{ and }
            A \subseteq B
        }.
\end{equation*}
%
In this case we also have $\mu^*(A) = \mu(A)$ if $A \in \calJ$.

\begin{definition}
    Let $m$ and $M$ be Borel measures on $X$. We say that $m$ is the \emph{essential measure} associated with $M$ if
    %
    \begin{equation*}
        % \label{eq:essential-measure-definition}
        m(A)
            = \sup \set{ M^*(B) }{ B \subseteq A \text{ and } M^*(B) < \infty },
    \end{equation*}
    %
    for all $A \in \borel{X}$.
\end{definition}


\begin{lemma}
    If $m$ and $M$ are Borel measures on $X$ and $m$ is the essential measure associated with $M$, then $m(A) = M(A)$ when $M(A) < \infty$ or $m(A) = \infty$, and
    %
    \begin{equation}
        \label{eq:essential-measure-equivalent-formula}
        m(A)
            = \sup \set{ M(B) }{ B \in \borel{X}, B \subseteq A \text{ and } M(B) < \infty },
    \end{equation}
    %
    for all $A \in \borel{X}$.
\end{lemma}

\begin{proof}
    Let $A \in \borel{X}$, and assume that $M(A) < \infty$. Then $m(A) = M^*(A) = M(A) < \infty$. For $\epsilon > 0$ there exists a $B \subseteq A$ such that
    %
    \begin{equation*}
        m(A)
            \leq M^*(B) + \epsilon
            \leq M(A) + \epsilon,
    \end{equation*}
    %
    and since $\epsilon$ was arbitrary, it follows that $m(A) \leq M(A)$. Conversely,
    %
    \begin{equation*}
        m(A)
            \geq M^*(A)
            = M(A),
    \end{equation*}
    %
    so in total $m(A) = M(A)$. With this \eqref{eq:essential-measure-equivalent-formula} is obvious when $M(A) < \infty$.

    Now assume that $m(A) = \infty$. Then for any $R > 0$ there exists a $B \subseteq A$ such that $M^*(B) \geq R$. Now let $C \in \borel{X}$ such that $B \subseteq C$ and $M(C) < \infty$. Then $B \subseteq A \intersect C \subseteq A$, so
    %
    \begin{equation*}
        R
            \leq M^*(B)
            \leq M(A \intersect C)
            \leq m(A).
    \end{equation*}
    %
    Since $R$ was arbitrary, $M$ can take on arbitrarily large but finite values on subsets of $A$, so \eqref{eq:essential-measure-equivalent-formula} follows. It also follows that $M(A) = \infty$.
\end{proof}


\chapter{Radon measures}

Let $X$ be a Hausdorff topological space.

\begin{definition}[Radon measures, $R_1$]
    A Radon measure on $X$ is a pair of measures $(M,m)$ on $\borel{X}$ such that
    %
    \begin{enumdef}
        \item $m$ is the essential measure associated with $M$,

        \item $M$ is locally finite and outer regular,

        \item $m$ is inner regular, and

        \item $m(B) = M(B)$ for $B \in \borel{X}$ if $B$ is open or $M(B) < \infty$.
    \end{enumdef}
\end{definition}

\begin{definition}[Radon measures, $R_2$]
    A Radon measure on $X$ is a measure $M$ on $\borel{X}$ that is locally finite, outer regular, and inner regular on open sets.
\end{definition}

\begin{definition}[Radon measures, $R_3$]
    A Radon measure on $X$ is a measure $m$ on $\borel{X}$ that is locally finite and inner regular.
\end{definition}


\end{document}