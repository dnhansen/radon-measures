% Document setup
\documentclass[article, a4paper, 11pt, oneside]{memoir}
\usepackage[utf8]{inputenc}
\usepackage[T1]{fontenc}
\usepackage[UKenglish]{babel}

% Document info
\newcommand\doctitle{Radon Measures}
\newcommand\docauthor{Danny Nygård Hansen}

% Formatting and layout
\usepackage[autostyle]{csquotes}
\usepackage[final]{microtype}
\usepackage{xcolor}
\frenchspacing
\usepackage{latex-sty/articlepagestyle}
\usepackage{latex-sty/articlesectionstyle}

% Fonts
\usepackage[largesmallcaps,partialup]{kpfonts}
\DeclareSymbolFontAlphabet{\mathrm}{operators} % https://tex.stackexchange.com/questions/40874/kpfonts-siunitx-and-math-alphabets
\linespread{1.06}
\let\mathfrak\undefined
\usepackage{eufrak}
\usepackage{inconsolata}
\usepackage{amssymb}

% Hyperlinks
\usepackage{hyperref}
\definecolor{linkcolor}{HTML}{4f4fa3}
\hypersetup{%
	pdftitle=\doctitle,
	pdfauthor=\docauthor,
	colorlinks,
	linkcolor=linkcolor,
	citecolor=linkcolor,
	urlcolor=linkcolor,
	bookmarksnumbered=true
}

% Equation numbering
\numberwithin{equation}{chapter}

% Footnotes
\footmarkstyle{\textsuperscript{#1}\hspace{0.25em}}

% Mathematics
\usepackage{latex-sty/basicmathcommands}
\usepackage{latex-sty/framedtheorems}
\usepackage{latex-sty/topologycommands}
\usepackage{tikz-cd}
\usetikzlibrary{babel}

% Lists
\usepackage{enumitem}
\setenumerate[0]{label=\normalfont(\arabic*)}

% Bibliography
\usepackage[backend=biber, style=authoryear, maxcitenames=2, useprefix]{biblatex}
\addbibresource{references.bib}

% Title
\title{\doctitle}
\author{\docauthor}


% Section style -- add to section style .sty?
\setsubsecheadstyle{\normalfont\itshape}


% Preimage -- to be added to mathcommands .sty
\newcommand{\preim}{^{-1}}


\newcommand{\calN}{\mathcal{N}}
\DeclarePairedDelimiter{\nhoodfilteraux}{(}{)}
% \newcommand{\nhoodfilter}[1]{\calN\nhoodfilteraux{#1}}
\newcommand{\nhoodfilter}[1]{\calN_{#1}}


\newcommand{\calU}{\mathcal{U}}
\newcommand{\calV}{\mathcal{V}}
\newcommand{\calW}{\mathcal{W}}
\newcommand{\calT}{\mathcal{T}}
\newcommand{\calB}{\mathcal{B}}
\newcommand{\calE}{\mathcal{E}}

\newcommand{\borel}[1]{\calB(#1)}
\DeclareMathOperator{\supp}{supp}

\begin{document}

\maketitle

\chapter{Introduction}


\chapter{General properties of measures}

We assume that the reader is familiar with abstract measure spaces and topological spaces. Below we fix terminology and prove some elementary results.


\section{Essential measures}

\newcommand{\calJ}{\mathcal{J}}
\newcommand{\powerset}[1]{2^{#1}}

If $\calJ \subseteq \powerset{X}$ and $\mu \colon \calJ \to [0,\infty]$ such that $\emptyset \in \calJ$ and $\mu(\emptyset) = 0$, then $\mu$ gives rise to an outer measure $\mu^*$ on $X$ by
%
\begin{equation*}
    \mu^*(A)
        = \inf \set[\Big]{
            \sum_{n=1}^\infty \mu(B_n)
        }{
            (B_n)_{n \in \naturals} \subseteq \calJ
            \text{ and }
            A \subseteq \bigunion_{n \in \naturals} B_n
        }.
\end{equation*}
%
In the case that $\mu$ is a measure and $\calJ$ is a $\sigma$-algebra, we may rephrase this as
%
\begin{equation*}
    \mu^*(A)
        = \inf \set{
            \mu(B)
        }{
            B \in \calJ
            \text{ and }
            A \subseteq B
        }.
\end{equation*}
%
In this case we also have $\mu^*(A) = \mu(A)$ if $A \in \calJ$.

\begin{definition}[Essential measures]
    Let $(X, \calE, M)$ be a measure space. We say that a map $m \colon \borel{X} \to [0,\infty]$ is the \emph{essential measure} associated with $M$ if
    %
    \begin{equation*}
        % \label{eq:essential-measure-definition}
        m(A)
            = \sup \set{ M^*(B) }{ B \subseteq A \text{ and } M^*(B) < \infty },
    \end{equation*}
    %
    for all $A \in \calE$.
\end{definition}


\begin{lemma}
    \label{thm:essential-measure-equivalent-formula}
    If $(X,\calE,M)$ is a measure space and $m$ is the essential measure associated with $M$, then $m(A) = M(A)$ when $M(A) < \infty$ or $m(A) = \infty$, and
    %
    \begin{equation}
        \label{eq:essential-measure-equivalent-formula}
        m(A)
            = \sup \set{ M(B) }{ B \in \calE, B \subseteq A \text{ and } M(B) < \infty },
    \end{equation}
    %
    for all $A \in \calE$. Furthermore, $m$ is a measure on $\calE$.
\end{lemma}

\begin{proof}
    Let $A \in \calE$, and assume that $M(A) < \infty$. Then $M^*(A) = M(A) < \infty$ by the definition of the outer measure $M^*$, so $m(A) = M^*(A)$ by the definition of $m$. With this, \eqref{eq:essential-measure-equivalent-formula} is obvious when $M(A) < \infty$.

    Now assume that $m(A) = \infty$. Then for any $R > 0$ there exists a $B \subseteq A$ such that $M^*(B) \geq R$. Now let $C \in \calE$ such that $B \subseteq C$ and $M(C) < \infty$. Then $B \subseteq A \intersect C \subseteq A$, so
    %
    \begin{equation*}
        R
            \leq M^*(B)
            \leq M(A \intersect C)
            < \infty.
    \end{equation*}
    %
    Since $R$ was arbitrary, $M$ can take on arbitrarily large but finite values on subsets of $A$, so \eqref{eq:essential-measure-equivalent-formula} follows. It also follows that $M(A) = \infty$.

    Finally we show that $m$ is a Borel measure on $X$. Let $(A_n)_{n \in \naturals}$ be a sequence of pairwise disjoint sets in $\calE$, and let $B \subseteq \bigunion_{n \in \naturals} A_n$ be such that $M(B) < \infty$. If we let $B_n = B \intersect A_n$, then
    %
    \begin{equation*}
        M(B)
            = M \Bigl( \bigunion_{n \in \naturals} B_n \Bigr)
            = \sum_{n=1}^\infty M(B_n)
            \leq \sum_{n=1}^\infty m(A_n),
    \end{equation*}
    %
    where the inequality follows since $B_n \subseteq A_n$ and $M(B_n) < \infty$. This inequality holds for all $B \subseteq \bigunion_{n \in \naturals} A_n$, so taking the supremum over such $B$ yields
    %
    \begin{equation*}
        m \Bigl( \bigunion_{n \in \naturals} A_n \Bigr)
            \leq \sum_{n=1}^\infty m(A_n).
    \end{equation*}
    %
    We prove the opposite inequality. If the left-hand side is infinite this is obvious, so assume that it is finite. For $\epsilon > 0$ there is then a sequence $(C_n)_{n \in \naturals}$ in $\calE$ with $C_n \subseteq A_n$ and $M(C_n) < \infty$ such that $m(A_n) \leq M(C_n) + \epsilon/2^n$. It follows that
    %
    \begin{equation*}
        \sum_{n=1}^\infty m(A_n)
            \leq \sum_{n=1}^\infty M(C_n) + \epsilon
            \leq \sum_{n=1}^\infty m(A_n) + \epsilon,
    \end{equation*}
    %
    and since $\epsilon$ was arbitrary, the inequality follows. [Something's missing here?]
\end{proof}


\section{Borel measures on Hausdorff spaces}

Below we let $X$ denote a Hausdorff topological space. A \emph{Borel measure} on $X$ is a measure on the Borel $\sigma$-algebra $\borel{X}$ of $X$. A Borel measure $\mu$ on $X$ is called \emph{outer regular} on a set $B \in \borel{X}$ if
%
\begin{equation*}
    \mu(B)
        = \inf \set{\mu(U)}{U \supseteq B, \text{$U$ open}},
\end{equation*}
%
and \emph{inner regular} on $B$ if
%
\begin{equation*}
    \mu(B)
        = \sup \set{\mu(K)}{K \subseteq B, \text{$K$ compact}}.
\end{equation*}
%
If $\mu$ is outer (inner) regular on all Borel sets, then we call it \emph{outer (inner) regular}. Furthermore, if $\mu$ is both outer and inner regular, then it is simply called \emph{regular}.

A Borel measure $\mu$ on $X$ is called \emph{locally finite} if every point has a neighbourhood $U$ with $\mu(U) < \infty$. We have the following characterisation of local finiteness:

\begin{proposition}
    \label{thm:local-finiteness-compacts}
    If a Borel measure on $X$ is locally finite, then it is finite on all compact sets. The converse is also true if $X$ is locally compact.
\end{proposition}

\begin{proof}
    Let $\mu$ be a locally finite Borel measure on $X$, and let $K \subseteq X$ be compact. Every $x \in K$ has an open neighbourhood $U_x$ with $\mu(U_x) < \infty$. The collection $\set{U_x}{x \in K}$ is an open cover of $K$, so it has a finite subcover, say $U_{x_1}, \ldots, U_{x_n}$. But then
    %
    \begin{equation*}
        \mu(K)
            \leq \mu \Bigl( \bigunion_{i=1}^n U_{x_i} \Bigr)
            \leq \sum_{i=1}^n \mu(U_{x_i})
            < \infty,
    \end{equation*}
    %
    as desired.

    Conversely, suppose that $X$ is locally compact and that $\mu$ is a Borel measure on $X$ that is finite on compact sets. Then every point has a compact neighbourhood, so every point has a neighbourhood on which $\mu$ is finite. Hence $\mu$ is locally finite.
\end{proof}

Above we defined 



\chapter{Radon measures}

Let $X$ be a Hausdorff topological space.

\begin{definition}[Radon measures, $R_1$]
    A Radon measure on $X$ is a pair of measures $(M,m)$ on $\borel{X}$ such that
    %
    \begin{enumdef}
        \item $m$ is the essential measure associated with $M$,

        \item $M$ is locally finite and outer regular,

        \item $m$ is inner regular, and

        \item $m(B) = M(B)$ for $B \in \borel{X}$ if $B$ is open or $M(B) < \infty$.
    \end{enumdef}
\end{definition}

\begin{definition}[Radon measures, $R_2$]
    A Radon measure on $X$ is a measure $M$ on $\borel{X}$ that is locally finite, outer regular, and inner regular on open sets.
\end{definition}

\begin{definition}[Radon measures, $R_3$]
    A Radon measure on $X$ is a measure $m$ on $\borel{X}$ that is locally finite and inner regular.
\end{definition}


\begin{theorem}
    \begin{enumthm}
        \item Let $(M,m)$ be an $R_1$-Radon measure on $X$. Then $M$ is an $R_2$-Radon measure, and $m$ is an $R_3$-Radon measure.

        \item Let $M$ be an $R_2$-Radon measure on $X$. If $m$ is the essential measure associated with $M$, then $(M,m)$ is an $R_1$-Radon measure.

        \item Let $m$ be an $R_3$-Radon measure on $X$, and define a map $M \colon \borel{X} \to [0,\infty]$ by
        %
        \begin{equation*}
            M(A)
                = \inf \set{m(U)}{\text{$U$ open and $A \subseteq U$}}.
        \end{equation*}
        %
        Then $(M,m)$ is an $R_1$-Radon measure on $X$.
    \end{enumthm}
\end{theorem}

\begin{proof}
    
\end{proof}

\newcommand{\calM}{\mathcal{M}}
\newcommand{\radonout}{\calM^{+}}
\newcommand{\radonin}{\calM^{-}}

Put this in different terms:

\begin{definition}[Radon measures]
    An \emph{outer Radon measure} on $X$ is a Borel measure on $X$ that is locally finite, outer regular, and inner regular on open sets.

    An \emph{inner Radon measure} on $X$ is a Borel measure on $X$ that is locally finite and inner regular.
\end{definition}
%
We denote the set of outer Radon measures on $X$ by $\radonout(X)$ and the set of inner Radon measures by $\radonin(X)$. Define maps $E \colon \radonout(X) \to \radonin(X)$ and $P \colon \radonin(X) \to \radonout(X)$, where $E$ maps an outer Radon measure to the corresponding essential measure, and $P$ maps an inner Radon measure to the corresponding principal measure.


\begin{lemma}
    \label{thm:outer-Radon-inner-regular-on-finites}
    Every outer Radon measure is inner regular on all its $\sigma$-finite sets.
\end{lemma}

\begin{proof}
    \textcite[Proposition~7.5]{folland2007}. [Folland assumes local compactness, is this a problem? If it is, Schwartz has a proof in the finite case which is all I need.]
\end{proof}

% Add to framedtheorems.sty
\newcommand{\mylistlabelfont}[1]{{\normalfont\color{linkcolor}\textit{#1}:}}
\newlist{proofsec}{description}{1}
\setlist[proofsec]{leftmargin=0pt, parsep=0pt, listparindent=\parindent, font=\mylistlabelfont}

\begin{theorem}
    The maps $E$ and $P$ are well-defined and each other's inverses.
\end{theorem}

\begin{proof}
\begin{proofsec}
    \item[Well-definition of $E$]
    Let $\mu$ be an outer Radon measure on $X$, and let $\nu$ be the associated essential measure. If $A \in \borel{X}$ and $\mu(A) < \infty$, then $\mu$ is inner regular on $A$ by \cref{thm:outer-Radon-inner-regular-on-finites}. But since $\mu$ and $\nu$ agree on sets with finite $\mu$-measure, $\nu$ is also inner regular on $A$.

    Similarly, every point in $X$ has a neighbourhood with finite $\mu$-measure, so this neighbourhood also has finite $\nu$-measure. Thus $\nu$ is an inner Radon measure.

    \item[Well-definition of $P$]
    Let $\nu$ be an inner Radon measure on $X$ with associated principal measure $\mu$. We first show that $\mu$ and $\nu$ agree on $\mu$-finite sets, so let $A \in \borel{X}$ with $\mu(A) < \infty$. Clearly $\nu(A) \leq \mu(A)$, so we prove the other inequality. Let $U \supseteq A$ be an open set with $\nu(U) < \infty$. Then also $\nu(U \setminus A) < \infty$, so for $\epsilon > 0$ there exists a compact set $K \subseteq U \setminus A$ with $\nu(U \setminus A) \leq \nu(K) + \epsilon$ by inner regularity. Then $V = U \setminus K$ is an open set containing $A$, so
    %
    \begin{equation*}
        \mu(A)
            \leq \nu(V)
            = \nu(A) + \nu(U \setminus A) - \nu(K)
            \leq \nu(A) + \epsilon.
    \end{equation*}
    %
    Since $\epsilon$ was arbitrary, it follows that $\mu(A) \leq \nu(A)$.

    Next we show that $\mu$ is $\sigma$-additive, so let $(A_n)_{n \in \naturals}$ be a sequence of disjoints sets in $\borel{X}$. For $\epsilon > 0$ there exists a sequence $(U_n)_{n \in \naturals}$ of open sets with $A_n \subseteq U_n$ such that $\nu(U_n) \leq \mu(A_n) + \epsilon$. It follows that
    %
    \begin{equation*}
        \mu \Bigl( \bigunion_{n \in \naturals} A_n \Bigr)
            \leq \nu \Bigl( \bigunion_{n \in \naturals} U_n \Bigr)
            = \sum_{n=1}^\infty \nu(U_n)
            \leq \sum_{n=1}^\infty \mu(A_n) + \epsilon.
    \end{equation*}
    %
    So $\mu$ is countably subadditive since $\epsilon$ was arbitrary. The opposite inequality is obvious if $\mu( \bigunion_{n \in \naturals} A_n ) = \infty$, and if not then the sets $A_n$ also have finite $\mu$-measure. Hence
    %
    \begin{equation*}
        \mu \Bigl( \bigunion_{n \in \naturals} A_n \Bigr)
            = \nu \Bigl( \bigunion_{n \in \naturals} A_n \Bigr)
            = \sum_{n=1}^\infty \nu(A_n)
            = \sum_{n=1}^\infty \mu(A_n).
    \end{equation*}
    %
    Thus $\mu$ is in fact a measure.

    Finally, $\mu$ is clearly locally finite since $\nu$ is, and since $\mu$ and $\nu$ agree on $\mu$-finite sets, outer regularity follows easily from the definition of $\mu$.

    \item[$P \circ E = \id$]
    Let $\mu$ be an outer Radon measure, $\nu = E(\mu)$ its corresponding essential measure, and $\mu' = P(\nu)$ the principal measure associated with $\nu$. We must show that $\mu = \mu'$.

    Let $A \in \borel{X}$. Since $\mu$ is outer regular,
    %
    \begin{equation*}
        \mu(A)
            = \inf \set{\mu(U)}{\text{$U$ open and $A \subseteq U$}}.
    \end{equation*}
    %
    Comparing this with the definition of $\mu'$ and recalling that $\nu \leq \mu$, we find that $\mu'(A) \leq \mu(A)$.

    For the opposite inequality, let $\epsilon > 0$ and let $U \supseteq A$ be an open set such that $\nu(U) \leq \mu'(A) + \epsilon$. Because $\nu(U) = \mu(U)$ by \cref{thm:Radon-agree-on-open-sets}, we have
    %
    \begin{equation*}
        \mu'(A) + \epsilon
            \geq \nu(U)
            = \mu(U)
            \geq \mu(A),
    \end{equation*}
    %
    so it follows that $\mu'(A) \geq \mu(A)$ since $\epsilon$ was arbitrary.

    \item[$E \circ P = \id$]
    Conversely, let $\nu$ be an inner Radon measure and let $\mu = P(\nu)$ and $\nu' = E(\mu)$. Let $A \in \borel{X}$ and notice that
    %
    \begin{align*}
        \nu(A)
            &= \sup \set{\nu(K)}{ \text{$K$ compact and $K \subseteq A$} } \\
            &= \sup \set{\mu(K)}{ \text{$K$ compact and $K \subseteq A$} } \\
            &= \sup \set{\mu(B)}{ \text{$B \in \borel{X}$, $B \subseteq A$ and $\mu(B) < \infty$} } \\
            &= \nu'(A).
    \end{align*}
    %
    The second and third equalities follow since $\mu$ is locally finite, hence finite on compact sets by \cref{thm:local-finiteness-compacts}, so $\nu(K) = \mu(K) < \infty$.
\end{proofsec}
\end{proof}



Now let $X$ be a locally compact Hausdorff space, and let $C_c(X)$ denote the space of continuous complex-valued functions on $X$. A linear functional $I$ on $C_c(X)$ is said to be \emph{positive} if $I(f) \geq 0$ when $f \geq 0$. A Borel measure $\mu$ on $X$ is called a \emph{representing measure} for $I$ if $I(f) = \int f \dif\mu$ for all $f \in C_c(X)$.

\begin{theorem}[The Riesz Representation Theorem]
    Every positive linear functional on $C_c(X)$ has a unique $R_2$-Radon representing measure.
\end{theorem}

\begin{proof}
    \textcite[Theorem~7.2]{folland2007}.
\end{proof}

\begin{proposition}
    Let $(M,m)$ be an $R_1$-Radon measure on $X$, and let $I$ be a positive linear functional on $C_c(X)$. Then $M$ is a representing measure for $I$ if and only if $m$ is. [Uniqueness?]
\end{proposition}

\begin{proof}
    This amounts to showing that
    %
    \begin{equation*}
        \int f \dif m = \int f \dif M
    \end{equation*}
    %
    for all $f \in C_c(X)$. Pick one such $f$, and let $K = \supp f$. Since $K$ is compact and both $m$ and $M$ are locally finite, \cref{thm:essential-measure-equivalent-formula} implies that $m$ and $M$ agree when restricted to $K$. The claim follows.
\end{proof}


Don't know where to put this, but just to write it down:

\begin{definition}[Principal measures]
    Let $m$ be a measure on a topological space $X$. The \emph{principal measure}\footnotemark{} associated with $m$ is the map $M \colon \borel{X} \to [0,\infty]$ given by
    %
    \begin{equation*}
        M(A)
            = \inf \set{m(U)}{\text{$U$ open and $A \subseteq U$}}.
    \end{equation*}
\end{definition}
\footnotetext{I do not believe this is standard terminology. \textcite{bauer2001} uses the term \textquote{principal measure} for the $R_2$-Radon measure determined by a positive linear functional, as in Riesz' representation theorem.}

[Is this a measure in general?]

\begin{lemma}
    \label{thm:Radon-agree-on-open-sets}
    Let $M$ be an $R_2$-Radon measure on $X$, and let $m$ be the essential measure associated with $M$. Then $m(U) = M(U)$ for all open $U \subseteq X$.
\end{lemma}

\begin{proof}
    Let $U \subseteq X$ be open. Since $m \leq M$ in general, we only need to show that $m(U) \geq M(U)$. Since $M$ is inner regular on $U$, there exists a sequence $(K_n)_{n \in \naturals}$ of compact subsets of $U$ such that $M(U) = \lim_{n \to \infty} M(K_n)$. Furthermore, $M(K_n) < \infty$ because $M$ is locally finite, so since $m$ and $M$ agree when $M$ is finite it follows that
    %
    \begin{equation*}
        M(U)
            = \lim_{n \to \infty} M(K_n)
            = \lim_{n \to \infty} m(K_n)
            \leq m(U),
    \end{equation*}
    %
    as desired.
\end{proof}

\begin{proposition}
    Let $M$ be an $R_2$-Radon measure on $X$, let $m$ be the essential measure associated with $M$, and let $M'$ be the principal measure associated with $m$. Then $M = M'$. In particular, every $R_2$-Radon measure is the principal measure of some $R_3$-Radon measure.
\end{proposition}

\begin{proof}
    Let $A \in \borel{X}$. Since $M$ is outer regular,
    %
    \begin{equation*}
        M(A)
            = \inf \set{M(U)}{\text{$U$ open and $A \subseteq U$}}.
    \end{equation*}
    %
    Comparing this with the definition of $M'$ and recalling that $m \leq M$, we find that $M'(A) \leq M(A)$.

    For the opposite inequality, let $\epsilon > 0$ and let $U \supseteq A$ be an open set such that $m(U) \leq M'(A) + \epsilon$. Because $m(U) = M(U)$ by [lemma], we have
    %
    \begin{equation*}
        M'(A) + \epsilon
            \geq m(U)
            = M(U)
            \geq M(A),
    \end{equation*}
    %
    so it follows that $M'(A) \geq M(A)$ since $\epsilon$ was arbitrary.
\end{proof}


\nocite{*}

\printbibliography


\end{document}